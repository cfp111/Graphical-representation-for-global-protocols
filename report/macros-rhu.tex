\newcommand{\MYPARAGRAPH}[1]{\paragraph{\textnormal{\textbf{#1}}}}

\definecolor{dkblue}{rgb}{0,0.1,0.5}
\definecolor{dkgreen}{rgb}{0,0.4,0}
\definecolor{dkred}{rgb}{0.4,0,0}


\newcommand{\CODESIZE}{\small}
%\newcommand{\CODESIZE}{\smaller}
\newcommand{\CODESTYLE}{\ttfamily}

\lstset%
{%
	%breakautoindent=true,
	%breaklines=true,
	captionpos=b,
	%columns=fixed,
	columns=flexible, % doesn't give proper space sizes, but fixed is much too fat
	%commentstyle=\itshape\CODESIZE\color{purple},
	commentstyle=\CODESTYLE\footnotesize\color{purple},
	escapeinside={*@}{@*},
	%extendedchars=true,
	float=hbp,
	frame=none,
	% identifierstyle=\color{black},
	language=Java,
	mathescape=true,
	numbers=none, % OLD: Numbers "on" by default: listings inside e.g. tabulars don't seem to work if we give parameters. % Maybe inline \lstset will help?
	numberstyle=\tiny,
	showspaces=false,
	showstringspaces=false,
	showtabs=false,
	stringstyle=\color{teal},
	tabsize=2%, % For some reason, has to be size 9 (equivalently 9 manual spaces) to match \noindent\CODE{12345678}. % maybe due to columns setting
	%xleftmargin=\DEFAULTXLEFTMARGIN % Roughly following the default tabsize.
}

\newcommand{\CODE}[1]{\texttt{\CODESIZE#1}} % Inline code listing (no additional formatting). convention: no colour formatting for inline code.
\newcommand{\GREYCODE}[1]{\CODE{
	%\color{gray}{#1}}} % Problem: braces aren't the same style
	\color{black}{#1}}} % Problem: braces aren't the same style
\newcommand{\SJCODE}[1]
{%
	\lstinline[style=SJ]+#1+%
}

\lstnewenvironment{CODELISTING}% % Listing without numbers or keywords; other parameters as above.
{%
	\onehalfspacing
	%\singlespacing
	\lstset%
	{%
		basicstyle=\CODESTYLE\footnotesize,
		keywordstyle=\CODESTYLE\footnotesize,
		%numbers=none,
		%tabsize=4 % Trying to keep tabsize a variable parameter isn't compatible with indentation conventions. The tabsize is now different to the SJBNF listing, but keep the default xleftmargin to align with BNF listings.
	}
}%
{%
	\doublespacing % Based on the settings of icthesis.sty
}

\lstdefinestyle{SJ}%
{%
	basicstyle   = \CODESTYLE\footnotesize,
	keywordstyle=[1]{
	%\bfseries
	\!\!\!\color{dkblue}
	\CODESTYLE\footnotesize},
	keywordstyle=[2]{
	%\bfseries
	\!\!\!\color{dkgreen}
	\CODESTYLE\footnotesize},
	%numbers     = none,
	%tabsize      = 4, % Same as the tabsize in the CODELISTING environment.
        moredelim=*[s][\footnotesize\color{dkgreen}]{<}{>},
        morekeywords =
	[1]{%
		%terminal, non, extend, RESULT
		protocol, role, choice, at, or, from, to, rec, parallel,
                and, interruptible, by, finish, continue, global, local, self, within
	},
	morekeywords =
	[2]{%
                int,Data,
	%	noalias,
	%	protocol,
	%	cbegin, sbegin, rec, end,
	%	using,
	%	accept, request,
	%	send, receive, receiveInt, receiveDouble, receiveFloat, receiveByte, receiveLong,
	%	outbranch, inbranch, outwhile, inwhile, recursion, recurse,
	%	spawn,
	%	typecase, when,
	%	register, select
	},
       literate={>=}{$\geq\ $}{2}{<=}{$\leq\ $}{2}
}

\lstnewenvironment{SJLISTING}%
{
	\lstset{style=SJ}
}
{
}


%%
% General latex stuff
%
\newcommand{\REF}[1]{\S\,\ref{#1}}

%%
% General code stuff
%
%\newcommand{\CODE}[1]{{\small \texttt{#1}}}
\newcommand{\OPASSIGN}{\, \CODE{:=} \,}
%\newcommand{\OPEQ}{\, \CODE{==} \,}
\newcommand{\OPEQ}{\ensuremath{=}}
\newcommand{\OPDEC}{\CODE{--}}
\newcommand{\OPINC}{\CODE{++}}
\newcommand{\OPSEQ}{\CODE{;}}
\newcommand{\OPNOT}{\ensuremath{\neg}}

%%
% General maths stuff
%\lstnewenvironment{SJLISTING}%
\newcommand{\OLINE}[1]{\ensuremath{\overline{#1}}}  % Overhead line
\newcommand{\SET}[1]{\ensuremath{\{ #1 \}}}         % Set braces
\newcommand{\FUN}[2]{\ensuremath{\mathsf{#1}(#2)}}  % Function call

%%
% General process stuff
%
\newcommand{\KWORD}[1]{\ensuremath{\mathsf{#1}}}
\newcommand{\DTYPE}[1]{\ensuremath{\mathtt{#1}}}  % Data type
\newcommand{\DVAL}[1]{\ensuremath{\mathtt{#1}}}   % Data value
\newcommand{\PPAR}{\ensuremath{\, | \,}}
\newcommand{\PPARGROUP}[1]{\ensuremath{(#1)}}
\newcommand{\PIF}{\ensuremath{\KWORD{if}}}
\newcommand{\PTHEN}{\ensuremath{\KWORD{then}}}
\newcommand{\PELSE}{\ensuremath{\KWORD{else}}}
\newcommand{\PNIL}{\ensuremath{\mathbf{0}}}
\newcommand{\EMPTY}{\ensuremath{\epsilon}}

%%
% General session types stuff
%
\newcommand{\LAB}[1]
	{\ensuremath{#1}}            % "Meta" label: "l"
\newcommand{\LABVAL}[1]
	{\ensuremath{\mathtt{#1}}}   % "Concrete" label: "quit"
\newcommand{\ROLE}[1]
	{\participant{#1}}           % Role/participant
\newcommand{\PARTY}[1]
	{\ensuremath{\mathsf{#1}}}   % Principal
\newcommand{\MSGlxS}[3]
	{\ensuremath{\MSGlx{#1}{#2\!:\!#3}}}  % lab(x:S)
\newcommand{\MSGlx}[2]
	{\ensuremath{#1 (#2)}}              % lab(x)  (variant needed to avoid ":")
\newcommand{\MSGlS}[2]
	{\MSGlx{#1}{#2}}                    % lab(S)
\newcommand{\BRANCH}[1]
	{\ensuremath{\SET{#1}}}         % Branch braces: {...}
\newcommand{\MUREC}[1]
	{\ensuremath{\mu \RECVAR{#1}}}  % Recursion prefix
\newcommand{\RECVAR}[1]
	{\ensuremath{\keyword{#1}}}     % Recursion var.

%%
% General session types stuff
%
\newcommand{\STATEVAR}[1]
	{\ensuremath{\mathtt{#1}}}                % State var.
\newcommand{\ROLEVAR}[2]
	{\ensuremath{\ROLE{#1} . \STATEVAR{#2}}}  % p.x
%\newcommand{\STATEDECL}[2]
	%{\ensuremath{#1 : \DTYPE{#2}}}
\newcommand{\STATEVARDECL}[2]
	%{\ensuremath{\STATEDECL{\STATEVAR{#1}}{#2}}}  % x : Nat
	{\ensuremath{\STATEVAR{#1} : \DTYPE{#2}}}      % x : Nat
\newcommand{\PARTYSTATEDECL}[2]
	{\ensuremath{\ROLE{#1}} : [#2]}                % p[x : Nat, ...]

%%
% Global types
%
\newcommand{\GSEP}{\ensuremath{.}}  % Global type separator
\newcommand{\GLOBAL}[1]
	{\ensuremath{\mathcal{#1}}}       % Global type name: G
\newcommand{\GLOBALi}[2]
	{\ensuremath{\GLOBAL{#1}_{#2}}}   % G_auction
\newcommand{\GSEND}[2]
	{\ensuremath{\ROLE{#1} \rightarrow \ROLE{#2} :}}  % p -> q:
%\newcommand{\GSEND}[5]
	%{\ensuremath{#1 \rightarrow #2 : \MSG{#3}{#4}{#5}}}

% Deprecated: use \BRANCH and \MUREC instead (same for both global and local)
\newcommand{\GBRA}[1]
	{\ensuremath{\SET{#1}}}         % Global branch braces: {...}
\newcommand{\GREC}[1]
	{\ensuremath{\mu \RECVAR{#1}}}  % Global recursion prefix

%%
% Global types for assertions/effects
%
\newcommand{\GLOBALDECL}[1]
	{\ensuremath{((#1))}}             % State var. decl. prefix
\newcommand{\LASS}[1]
	{\ensuremath{\langle #1 \rangle}}  % Left assertion: {...}
\newcommand{\LEFF}[1]
	{\LASS{#1}}                        % Left effect: {...}
\newcommand{\LASSEFF}[2]
	{\ensuremath{\LASS{#1,\, #2}}}     % Left assertion and effect: {..., ...}
\newcommand{\RASS}[1]
	{\ensuremath{\{ #1 \}}}            % Right assertion: <...>
\newcommand{\REFF}[1]
	{\RASS{#1}}                        % Right effect: <...>
\newcommand{\RASSEFF}[2]
	{\ensuremath{\RASS{#1,\, #2}}}     % Right assertion and effect: <..., ...>
\newcommand{\GRECtexA}[4]
	{\ensuremath{\MUREC{#1} \langle #2 \rangle (#3) \{ #4 \}}}
\newcommand{\GRECVARte}[2]
	{\ensuremath{\RECVAR{#1} \langle #2 \rangle}}

%%
% Local types
%
\newcommand{\LSEP}{\ensuremath{.}}  % Local type separator
\newcommand{\LOCAL}[1]
	{\ensuremath{\mathcal{#1}}}       % Local type name: L
\newcommand{\LOCALi}[2]
	{\ensuremath{\LOCAL{#1}_{#2}}}    % L_auction
\newcommand{\LSEND}[1]
	{\ensuremath{\ROLE{#1} \,!\,}}    % p!
\newcommand{\LRECV}[1]
	{\ensuremath{\ROLE{#1} \,?}}      % q!

%%
% Local types for assertions/effeacts
%
\newcommand{\LOCALDECL}[1]
	{\ensuremath{[#1]}}               % State var. decl. prefix
\newcommand{\LRECtexA}[4]
	{\GRECtexA{#1}{#2}{#3}{#4}}
\newcommand{\LRECVARte}[2]
	{\GRECVARte{#1}{#2}}

%%
% Session processes (1)
%
\newcommand{\POSEP}{\ensuremath{;}}  % After output prefixes
\newcommand{\PESEP}{\ensuremath{;}}  % Separator for effects
\newcommand{\PSEP}{\ensuremath{.}}   % Others: input prefixes, recursion, etc.
\newcommand{\PINIT}[1]
	{\ensuremath{\mathsf{#1}}}         % Initial process
\newcommand{\PINITi}[2]
	{\ensuremath{\mathsf{#1}_{#2}}}    % e.g. initial P_Server
%\newcommand{\PRUNT}
	%{\ensuremath{#1}}
\newcommand{\PREQ}[4]
	{\ensuremath{\OLINE{#1} \langle #2 [\ROLE{#3}] : \GLOBAL{#4} \rangle}}
                                                                % a~<x[p]:G>
\newcommand{\PACC}[4]
	{\ensuremath{#1 ( #2 [ \ROLE{#3} ] : \GLOBAL{#4} )}}          % a(x[p]:G)
\newcommand{\PSEND}[5]
	{\ensuremath{#1 [\ROLE{#2}, \ROLE{#3}] \,!\, \LAB{#4} \langle #5 \rangle}}
	                                                              % k[p,q]!l<e>
\newcommand{\PRECV}[3]
	{\ensuremath{#1 [\ROLE{#2}, \ROLE{#3}] \,?\,}}                % k[p,q]?
\newcommand{\PRECX}[1]
	{\ensuremath{\mu #1}}                            % mu X
\newcommand{\PRECXx}[2]
	{\ensuremath{\PRECX{#1} (#2)}}                   % mu X (x) ...
\newcommand{\PRECe}[1]
	{\ensuremath{\langle #1 \rangle}}                % ... <e>
\newcommand{\PRECVARX}[1]
	{\ensuremath{#1}}                                % X
\newcommand{\PRECVARXe}[2]
	{\ensuremath{\PRECVARX{#1} \langle #2 \rangle}}  % X<e>

% Deprecated: syntax change
\newcommand{\PRECx}[2]
	{\ensuremath{\MUREC{#1} (#2)}}
\newcommand{\PRECtx}[2]
	{\ensuremath{\MUREC{#1} (#2)}}      % mu t (x) ...
\newcommand{\PRECVARte}[2]
	{\GRECVARte{#1}{#2}}

%%
% Session processes (2)
%
\newcommand{\PNEWKW}{\KWORD{new}}
\newcommand{\PREGKW}{\KWORD{reg}}
\newcommand{\PINKW}{\KWORD{in}}
%\newcommand{\PWITHKW}{\KWORD{with}}
%\newcommand{\PINVITE}[2]
	%{\ensuremath{\mathtt{I}(\mathcal{#1}[\ROLE{#2}])}}          % I(G[p])
\newcommand{\PNEWs}[4]
	{\ensuremath{\PNEWKW\, (\AT{#1}{#2}, \ROLE{#3}) \,\PINKW\, #4}}  % new (s:G, p) in P
\newcommand{\PNEWa}[4]
	{\ensuremath{\PREGKW\, \AT{#1}{#2}[\ROLE{#3}] \,\PINKW\, #4}}
                                                        % new a:I(G[p]) in P
\newcommand{\PNEWp}[4]
	{\ensuremath{\PNEWKW\, \AT{#1}{#2} \,\KWORD{with}\, [#3] \,\PINKW\, #4}}
                                           % new \alpha:\Gamma with [P] in Q

% Deprecated: not needed
%\newcommand{\PJOINKW}{\KWORD{join}}
\newcommand{\PJOIN}[2]
	{\ensuremath{\KWORD{join}\, #1[#2]}}  % join s[p]

%%
% Session processes (3)
%
\newcommand{\PLOCK}
	{\ensuremath{\blacktriangledown \,}}
\newcommand{\PUNLOCK}
	{\ensuremath{\blacktriangle}}
\newcommand{\PGET}[2]
	{\ensuremath{#1 \OPASSIGN get(\STATEVAR{#2})}}  % x := get(f)
\newcommand{\PPUT}[2]
	{\ensuremath{put(#1, \STATEVAR{#2})}}           % put(f, e)
\newcommand{\PLRECV}[3]
	{#1 [\ROLE{#2}, \ROLE{#3}] \, ? \blacktriangledown}

%%
% Session processes (4)
%
\newcommand{\PSNET}[1]
	{\ensuremath{#1}}     % Static network: N
\newcommand{\PSNETi}[2]
	{\ensuremath{#1_{#2}}}
%\newcommand{\PDNET}[1]
	%{\ensuremath{\mathcal{#1}}}
\newcommand{\PICHAN}[2]
	{\ensuremath{\mathtt{I}(#1 [\ROLE{#2}])}}
\newcommand{\POCHAN}[2]
	{\ensuremath{\mathtt{O}(#1 [\ROLE{#2}])}}
\newcommand{\PNETQUEUE}[2]
	{\ensuremath{\langle #1 ; #2 \rangle}}

\newcommand{\RAYCOMMENT}[1]{~\\ \textbf{RAY:} #1}

